%class
\documentclass[a4paper, 12pt]{article}

%encoding
\usepackage[T1]{fontenc}           %font encoding
\usepackage[utf8]{inputenc}         %script encoding

%packages
\usepackage[czech]{babel}           %language
\usepackage[a4paper, text={17cm,24cm}, left=2cm, top=3 cm]{geometry}		%layout
\usepackage{times}					%font
\usepackage[ruled, czech, linesnumbered, longend, noline]{algorithm2e}		%algorithms
\usepackage[unicode,hidelinks]{hyperref}	%links
\usepackage{amsmath}
\usepackage{tabularx}
\usepackage{multicol}
\usepackage{multirow}
\usepackage{graphicx}
\usepackage{float}
\usepackage{csquotes}
\usepackage{xcolor}
\usepackage{caption}
\usepackage{fancyvrb}

\urlstyle{same}

\newcommand{\urlhref}[2]{\href{#1}{\textcolor{cyan}{\underline{#2}}}}
\newcommand{\iic}{I\textsuperscript{2}C }

\begin{document}

\begin{titlepage}
	\centering

	\vspace*{\stretch{0.382}}

	{\Huge Chytrá domácnost\\[0.4em]
		\LARGE BPC-IOT projekt \#7 2024}

	\vspace*{\stretch{0.618}}

	\begin{table}[H]
		\hfill
		\begin{tabularx}{0.3\textwidth}{Xr}
			David Tobolík & xtobol06 \\
			Name          & login    \\
		\end{tabularx}
	\end{table}
\end{titlepage}

\tableofcontents
\newpage

\section{Zadání}
\section{Popis aplikace}


\section{Volba technologií}
\subsection{Bezdrátová komunikace}
\subsection{Komunikační protokol}
\subsection{Napájení}


\subsection{Alternativní situace}
\textit{
	Jakým způsobem byste řešili senzory a aktory umístěné mimo vnitřní prostory
	domu (např. zavlažování, skleník atp.) v případě, že neexistuje outdoor pokrytí
	zvolenou technologií?
}

\end{document}

